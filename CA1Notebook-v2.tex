% Options for packages loaded elsewhere
\PassOptionsToPackage{unicode}{hyperref}
\PassOptionsToPackage{hyphens}{url}
%
\documentclass[
]{article}
\usepackage{amsmath,amssymb}
\usepackage{iftex}
\ifPDFTeX
  \usepackage[T1]{fontenc}
  \usepackage[utf8]{inputenc}
  \usepackage{textcomp} % provide euro and other symbols
\else % if luatex or xetex
  \usepackage{unicode-math} % this also loads fontspec
  \defaultfontfeatures{Scale=MatchLowercase}
  \defaultfontfeatures[\rmfamily]{Ligatures=TeX,Scale=1}
\fi
\usepackage{lmodern}
\ifPDFTeX\else
  % xetex/luatex font selection
\fi
% Use upquote if available, for straight quotes in verbatim environments
\IfFileExists{upquote.sty}{\usepackage{upquote}}{}
\IfFileExists{microtype.sty}{% use microtype if available
  \usepackage[]{microtype}
  \UseMicrotypeSet[protrusion]{basicmath} % disable protrusion for tt fonts
}{}
\makeatletter
\@ifundefined{KOMAClassName}{% if non-KOMA class
  \IfFileExists{parskip.sty}{%
    \usepackage{parskip}
  }{% else
    \setlength{\parindent}{0pt}
    \setlength{\parskip}{6pt plus 2pt minus 1pt}}
}{% if KOMA class
  \KOMAoptions{parskip=half}}
\makeatother
\usepackage{xcolor}
\usepackage[margin=1in]{geometry}
\usepackage{color}
\usepackage{fancyvrb}
\newcommand{\VerbBar}{|}
\newcommand{\VERB}{\Verb[commandchars=\\\{\}]}
\DefineVerbatimEnvironment{Highlighting}{Verbatim}{commandchars=\\\{\}}
% Add ',fontsize=\small' for more characters per line
\usepackage{framed}
\definecolor{shadecolor}{RGB}{248,248,248}
\newenvironment{Shaded}{\begin{snugshade}}{\end{snugshade}}
\newcommand{\AlertTok}[1]{\textcolor[rgb]{0.94,0.16,0.16}{#1}}
\newcommand{\AnnotationTok}[1]{\textcolor[rgb]{0.56,0.35,0.01}{\textbf{\textit{#1}}}}
\newcommand{\AttributeTok}[1]{\textcolor[rgb]{0.13,0.29,0.53}{#1}}
\newcommand{\BaseNTok}[1]{\textcolor[rgb]{0.00,0.00,0.81}{#1}}
\newcommand{\BuiltInTok}[1]{#1}
\newcommand{\CharTok}[1]{\textcolor[rgb]{0.31,0.60,0.02}{#1}}
\newcommand{\CommentTok}[1]{\textcolor[rgb]{0.56,0.35,0.01}{\textit{#1}}}
\newcommand{\CommentVarTok}[1]{\textcolor[rgb]{0.56,0.35,0.01}{\textbf{\textit{#1}}}}
\newcommand{\ConstantTok}[1]{\textcolor[rgb]{0.56,0.35,0.01}{#1}}
\newcommand{\ControlFlowTok}[1]{\textcolor[rgb]{0.13,0.29,0.53}{\textbf{#1}}}
\newcommand{\DataTypeTok}[1]{\textcolor[rgb]{0.13,0.29,0.53}{#1}}
\newcommand{\DecValTok}[1]{\textcolor[rgb]{0.00,0.00,0.81}{#1}}
\newcommand{\DocumentationTok}[1]{\textcolor[rgb]{0.56,0.35,0.01}{\textbf{\textit{#1}}}}
\newcommand{\ErrorTok}[1]{\textcolor[rgb]{0.64,0.00,0.00}{\textbf{#1}}}
\newcommand{\ExtensionTok}[1]{#1}
\newcommand{\FloatTok}[1]{\textcolor[rgb]{0.00,0.00,0.81}{#1}}
\newcommand{\FunctionTok}[1]{\textcolor[rgb]{0.13,0.29,0.53}{\textbf{#1}}}
\newcommand{\ImportTok}[1]{#1}
\newcommand{\InformationTok}[1]{\textcolor[rgb]{0.56,0.35,0.01}{\textbf{\textit{#1}}}}
\newcommand{\KeywordTok}[1]{\textcolor[rgb]{0.13,0.29,0.53}{\textbf{#1}}}
\newcommand{\NormalTok}[1]{#1}
\newcommand{\OperatorTok}[1]{\textcolor[rgb]{0.81,0.36,0.00}{\textbf{#1}}}
\newcommand{\OtherTok}[1]{\textcolor[rgb]{0.56,0.35,0.01}{#1}}
\newcommand{\PreprocessorTok}[1]{\textcolor[rgb]{0.56,0.35,0.01}{\textit{#1}}}
\newcommand{\RegionMarkerTok}[1]{#1}
\newcommand{\SpecialCharTok}[1]{\textcolor[rgb]{0.81,0.36,0.00}{\textbf{#1}}}
\newcommand{\SpecialStringTok}[1]{\textcolor[rgb]{0.31,0.60,0.02}{#1}}
\newcommand{\StringTok}[1]{\textcolor[rgb]{0.31,0.60,0.02}{#1}}
\newcommand{\VariableTok}[1]{\textcolor[rgb]{0.00,0.00,0.00}{#1}}
\newcommand{\VerbatimStringTok}[1]{\textcolor[rgb]{0.31,0.60,0.02}{#1}}
\newcommand{\WarningTok}[1]{\textcolor[rgb]{0.56,0.35,0.01}{\textbf{\textit{#1}}}}
\usepackage{graphicx}
\makeatletter
\def\maxwidth{\ifdim\Gin@nat@width>\linewidth\linewidth\else\Gin@nat@width\fi}
\def\maxheight{\ifdim\Gin@nat@height>\textheight\textheight\else\Gin@nat@height\fi}
\makeatother
% Scale images if necessary, so that they will not overflow the page
% margins by default, and it is still possible to overwrite the defaults
% using explicit options in \includegraphics[width, height, ...]{}
\setkeys{Gin}{width=\maxwidth,height=\maxheight,keepaspectratio}
% Set default figure placement to htbp
\makeatletter
\def\fps@figure{htbp}
\makeatother
\setlength{\emergencystretch}{3em} % prevent overfull lines
\providecommand{\tightlist}{%
  \setlength{\itemsep}{0pt}\setlength{\parskip}{0pt}}
\setcounter{secnumdepth}{-\maxdimen} % remove section numbering
\ifLuaTeX
  \usepackage{selnolig}  % disable illegal ligatures
\fi
\IfFileExists{bookmark.sty}{\usepackage{bookmark}}{\usepackage{hyperref}}
\IfFileExists{xurl.sty}{\usepackage{xurl}}{} % add URL line breaks if available
\urlstyle{same}
\hypersetup{
  pdftitle={R Notebook},
  hidelinks,
  pdfcreator={LaTeX via pandoc}}

\title{R Notebook}
\author{}
\date{\vspace{-2.5em}}

\begin{document}
\maketitle

This is an \href{http://rmarkdown.rstudio.com}{R Markdown} Notebook.
When you execute code within the notebook, the results appear beneath
the code.

Try executing this chunk by clicking the \emph{Run} button within the
chunk or by placing your cursor inside it and pressing
\emph{Ctrl+Shift+Enter}.

\begin{Shaded}
\begin{Highlighting}[]
\CommentTok{\# Import and view data}
\NormalTok{dataCA1Control }\OtherTok{\textless{}{-}} \FunctionTok{read.csv}\NormalTok{(}\StringTok{"GROUP\_3\_2023\_GCA\_RESULTS\_CONTROL.csv"}\NormalTok{)}
\FunctionTok{summary}\NormalTok{(dataCA1Control)}
\end{Highlighting}
\end{Shaded}

\begin{verbatim}
##    patientID        gender           pretrial_GAD    pretrial_STAI  
##  Min.   : 1.00   Length:90          Min.   :-16.00   Min.   :47.00  
##  1st Qu.:23.25   Class :character   1st Qu.: 14.00   1st Qu.:60.00  
##  Median :45.50   Mode  :character   Median : 15.00   Median :63.00  
##  Mean   :45.50                      Mean   : 14.62   Mean   :62.46  
##  3rd Qu.:67.75                      3rd Qu.: 17.00   3rd Qu.:65.00  
##  Max.   :90.00                      Max.   : 21.00   Max.   :72.00  
##  posttrial_GAD   posttrial_STAI  
##  Min.   : 6.00   Min.   : 43.00  
##  1st Qu.:13.00   1st Qu.: 60.00  
##  Median :15.00   Median : 64.00  
##  Mean   :14.71   Mean   : 64.07  
##  3rd Qu.:17.00   3rd Qu.: 67.75  
##  Max.   :23.00   Max.   :103.00
\end{verbatim}

\begin{Shaded}
\begin{Highlighting}[]
\CommentTok{\# Remove negative values}
\NormalTok{pretrial\_GAD }\OtherTok{\textless{}{-}} \FunctionTok{c}\NormalTok{(dataCA1Control}\SpecialCharTok{$}\NormalTok{pretrial\_GAD)}
\NormalTok{pretrial\_GAD}
\end{Highlighting}
\end{Shaded}

\begin{verbatim}
##  [1]  19  17  19  18  14  14  17  17  11  14  13  11  14  18  16  16  17  15  14
## [20]  14  15  13  17  12  18  15  17  10  14  16  19  20  19  16  16  14  19  15
## [39]  16  14  15  14   7  15  20  13  10  21  14  14  15  18  16  14   6  17  15
## [58]  15  13  15  16  14  15  14  13  14  17  19  12  14  15  21  13  16 -16  17
## [77]  12  14  10  13  12  15  15  14  12  19  13  15  12  16
\end{verbatim}

\begin{Shaded}
\begin{Highlighting}[]
\NormalTok{Pos\_pretrial\_GAD }\OtherTok{\textless{}{-}}\NormalTok{ pretrial\_GAD[pretrial\_GAD }\SpecialCharTok{\textgreater{}=} \DecValTok{0}\NormalTok{]}
\NormalTok{Pos\_pretrial\_GAD}
\end{Highlighting}
\end{Shaded}

\begin{verbatim}
##  [1] 19 17 19 18 14 14 17 17 11 14 13 11 14 18 16 16 17 15 14 14 15 13 17 12 18
## [26] 15 17 10 14 16 19 20 19 16 16 14 19 15 16 14 15 14  7 15 20 13 10 21 14 14
## [51] 15 18 16 14  6 17 15 15 13 15 16 14 15 14 13 14 17 19 12 14 15 21 13 16 17
## [76] 12 14 10 13 12 15 15 14 12 19 13 15 12 16
\end{verbatim}

\begin{Shaded}
\begin{Highlighting}[]
\CommentTok{\# Histogram of positive pretrial GAD values}
\FunctionTok{hist}\NormalTok{(Pos\_pretrial\_GAD)}
\end{Highlighting}
\end{Shaded}

\includegraphics{CA1Notebook-v2_files/figure-latex/unnamed-chunk-4-1.pdf}

\begin{Shaded}
\begin{Highlighting}[]
\CommentTok{\# Check data is numeric}
\FunctionTok{is.numeric}\NormalTok{(dataCA1Control)}
\end{Highlighting}
\end{Shaded}

\begin{verbatim}
## [1] FALSE
\end{verbatim}

\begin{Shaded}
\begin{Highlighting}[]
\CommentTok{\# Histogram of post{-}trial GAD}
\NormalTok{dataCA1Control }\OtherTok{\textless{}{-}} \FunctionTok{read.csv}\NormalTok{(}\StringTok{"GROUP\_3\_2023\_GCA\_RESULTS\_CONTROL.csv"}\NormalTok{)}
\NormalTok{posttrial\_GAD }\OtherTok{\textless{}{-}} \FunctionTok{c}\NormalTok{(dataCA1Control}\SpecialCharTok{$}\NormalTok{posttrial\_GAD)}
\NormalTok{posttrial\_GAD}
\end{Highlighting}
\end{Shaded}

\begin{verbatim}
##  [1] 18 19 19 17 16 11 18 14 11 16 15  9 14 18 17 13 18 14 16 15 15 13 16 13 19
## [26] 15 15 12 11 17 21 17 19 15 15 14 18 13 16 16 17 15  8 16 19 11 11 23 15 13
## [51] 16 18 13 12  6 16 14 17 13 15 14 15 14 12 12 15 18 17 14 11 14 19 13 14 17
## [76] 19 10 14  7 13 11 15 13 12 13 20 14 13  9 16
\end{verbatim}

\begin{Shaded}
\begin{Highlighting}[]
\FunctionTok{hist}\NormalTok{(posttrial\_GAD)}
\end{Highlighting}
\end{Shaded}

\includegraphics{CA1Notebook-v2_files/figure-latex/unnamed-chunk-6-1.pdf}

Add a new chunk by clicking the \emph{Insert Chunk} button on the
toolbar or by pressing \emph{Ctrl+Alt+I}.

When you save the notebook, an HTML file containing the code and output
will be saved alongside it (click the \emph{Preview} button or press
\emph{Ctrl+Shift+K} to preview the HTML file).

The preview shows you a rendered HTML copy of the contents of the
editor. Consequently, unlike \emph{Knit}, \emph{Preview} does not run
any R code chunks. Instead, the output of the chunk when it was last run
in the editor is displayed.

\end{document}
